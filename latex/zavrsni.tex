\documentclass[times, utf8, zavrsni]{fer}
\usepackage{booktabs}

\begin{document}

\thesisnumber{7031}

\title{Višedretvena izvedba algoritma računanja histograma orijentacija gradijenata u slikama}

\author{Mateo Imbrišak}

\maketitle

% Dodavanje zahvale ili prazne stranice. Ako ne želite dodati zahvalu, naredbu ostavite radi prazne stranice.
\zahvala{}

\tableofcontents

\chapter{Uvod}
Uporaba računalnog vida je danas jako raširena. Koristi se u sustavima za raspoznavanje objekata na slikama, prepoznavanje osoba, prebrojavanje pojedinih objekata na slici te kao pomoć pri orijentaciji autonomnih vozila. U svim područjima je poželjna veća brzina izvođenja no u nekima je upravo to ključna značajka. Glavni problem brzini izvođenja predstavlja činjenica da je proces dobivanje računalu korisnih podataka iz slike računski vrlo zahtjevan. Veliki broj modernih algoritama računalnog vida često izvodi identične i nezavisne operacije na velikom broju piksela pojedine slike te ih je zato moguće značajno ubrzati višedretvenom izvedbom. \\

Cilj ovog rada je ostvariti višedretvenu izvedbu algoritma računanja opisnika temeljenog na histogramu orijentacija gradijenata pomoću kojeg je moguće naučiti neki model da raspoznaje sadrži li neka slika traženi objekt ili ne. Iako se navedeni postupak može koristiti za detekciju proizvoljnih objekata u ovom radu koristi se prozor širine 64 i visine 128 pikseal, kao i u originalnom radu \citep{dalal2005histograms}, predviđen za detekciju pješaka. \\

Osnovni pojmovi, provlemi i prednosti višedretvenosti detaljnije se objašnjeni u drugom poglavlju. Treće poglavlje opisuje postupak računanja histograma orijentacija gradijenata u slikama. Opis detekcije objekata kao i model koji se koristi u te svrhe opisani su u četvrtom poglavlju. Peto poglavlje opisuje implementaciju programskog rješenja dok šesto analizira rezultate i performanse implementacije. Na kraju se nalazi zaključak te je dan popis korištene literature kao i sažetak na hrvatskom i negleskom jeziku.

\chapter{Višedretvena paralelizacija}

\chapter{Algoritam računanja histograma orijentacija gradijenata u slikama}

\chapter{Detekcija pješaka}

\chapter{Programsko rješenje}

\chapter{Rezultati}

\chapter{Zaključak}
Zaključak.

\bibliography{literatura}
\bibliographystyle{fer}

\begin{sazetak}
Sažetak na hrvatskom jeziku.

\kljucnerijeci{Ključne riječi, odvojene zarezima.}
\end{sazetak}

\engtitle{Multi-threaded Implementation of Algorithm for Calculation of Histogram of Oriented Gradients in Images}
\begin{abstract}
Abstract.

\keywords{Keywords.}
\end{abstract}

\end{document}
